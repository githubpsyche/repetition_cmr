I had to write up a study proposal for my course with Prof. SBS Spring 2021.

In the study of memory, one of the most widely documented mnemonic devices is the spacing effect: when repeated study of an item occurs space apart in time rather than massed together, later retrieval performance gets enhanced (Ebbinghaus, 1885). The effect has been observed across a variety of memory tasks such as recognition (Hintzman, 1976) and retrieval practice (Middleton et al, 2016), but is especially prominent under the free recall task where “the probability of recalling a repeated word often increases monotonically to spacings of 20 or more items” (Lohnas and Kahana, 2014).  In free recall tasks, research participants are presented with a sequence of items to study - usually word lists - and subsequently prompted to recall as many items from the list as possible in any order. 

Three main explanations are generally proposed to explain how spacing effects are realized with a free recall task. Under the contextual-variability account (Bower, 1972), each time an item is studied, it’s associated in memory with the current state of a slowly drifting contextual representation. Depending on how spaced apart two presentations of an item might be, the contextual states they are associated with might either be very similar or very distinct. Later, participants use the current state of their contextual representation to probe their memories and retrieve items during free recall. When an item has been associated with diverse contextual states, it can correspondingly be retrieved using diverse possible cues. In this way, the improvements in recall we gain from spacing presentations of an item are explained in terms of variation in the range of possible cues that can trigger recall of that item. 

A study-phase retrieval account of the spacing effect instead supposes that when we study a successively presented item, we retrieve memories of the repeated item’s earlier occurrences and their associated contexts. When this happens, it’s proposed that retrieved information is memorally associated with information corresponding to the current presentation context. To put it another way, while the contextual-variability account of the spacing effect emphasizes associating memory of an item with memories of multiple distinct contextual states, the study-phase retrieval account further draws an association between the contextual states themselves, creating an additional possible cue for retrieval whose distinctiveness increases with presentation spacing.

While the above accounts of spacing focus on variation in possible contextual cues, a third account of the spacing effects supposes that a difference in depth of processing between spaced and massed repetitions of an item explain the effect. Called the deficient-processing theory, it assumes that a repeated item’s memory trace is encoded (or retrieved) with strength inversely proportional to its lag. Explanations why can vary. Collins, Milliken, & Jamieson (2020) for example, focus on an interaction of familiarity and attention in a model of the phenomenon. “Biased exogenous attention” might favor unfamiliar, novel items over familiar ones leaving repetitions of items encoded less thoroughly. Spacing repetitions of items is thought to countervail this effect.

Under retrieved context theories of memory search such as the Context Maintenance and Retrieval (CMR) model (Polyn, Norman, & Kahana, 2009), representations of studied items in a free recall experiment are associated with an internal representation of temporal context that changes slowly during a study period. These associations in turn are thought to account for organizational effects in recall sequences, such as the tendency for related items to be recalled successively (temporal contiguity), or the enhancement in recall probabilities for items that occur early or later in a study list (the serial position effect) (Murdock, 1962). CMR and other retrieved context models have proven quite successful at accounting for a broad variety of memory phenomena, including some patterns tied to the spacing effect.

CMR effectively implements both the contextual variability and study-phase retrieval accounts of the spacing effect. Its emphasis on the role of item-to-context associations built during encoding in the organization of later recall realizes the core ideas of the contextual-variability account, while its mechanism of determining how internal representations of temporal context change over the course of study by retrieving contextual associations corresponding to the current item helps realize the study-phase retrieval account of the spacing effect. 

An analysis of CMR by Lohnas et al (2014) found that it could account for the broad trend of the spacing effect through the independent interaction of these two model features. They had a group of study participants perform four different variations of the free recall task. In one condition, each unique item was only presented and studied once before retrieval. In a massed repetition condition, each item was presented twice, with no spacing between each presentation (e.g. 1, 1, 2, 2, 3, 3, … rather than 1, 2, 3, 4, …). A third condition varied the amount of spacing between presentations while a fourth condition mixed elements of the three other conditions. Across several analyses, including the paradigmatic comparison of recall probability as a function of the space between repeated item presentations, CMR could realize the broad trends of all patterns studied.

Despite these successes, Lohnas et al’s simulations of CMR dramatically underpredicted the magnitude of the spacing effect that Lohnas et al (2014) observed in their study. Indeed, even when I replicate their simulations using a version of the model with parameters fit directly to their dataset, I found that CMR fails to identify a large increasing difference in recall probabilities for items with presentations spaced 1-2, 3-5, or 6-8 positions from one another. These results suggest that even as CMR realizes the encoding variability and study-phase retrieval accounts of the spacing effect, it cannot fully account for the size of the effect observable in actual behavioral data.

I propose creating and testing different modifications of CMR that implement the other deficient-processing account of the spacing effect along with the mechanisms it already supports. Lohnas et al (2014) proposes that deficient-processing as a function of repetition proximity could be implemented into the model by having the amount of context retrieved for a repeated item decreases as a function of its lag, but there are several different ways to realize this modification with different consequences for the model’s predictions about the the representations encoded during study, as well as course of memory search during retrieval. I propose replicating Lohnas et al’s experiments and performing a model comparison analysis to tease out which might best account for the results observed.
