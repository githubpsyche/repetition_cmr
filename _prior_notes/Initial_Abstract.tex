\citet{siegel2014retrieved} showed that retrieved context theory realizes two mechanisms commonly marshalled to explain spacing and related repetition effects in free recall:

Under the contextual-variability account \citep{anderson1972recognition}, each time an item is studied, it’s associated in memory with the current state of a slowly drifting contextual representation. Depending on how spaced apart two presentations of an item might be, the contextual states they are associated with might either be very similar or very distinct. Later, participants use the current state of their contextual representation to probe their memories and retrieve items during free recall. When an item has been associated with diverse contextual states, it can correspondingly be retrieved using diverse possible cues. In this way, the improvements in recall we gain from spacing presentations of an item are explained in terms of variation in the range of possible cues that can trigger recall of that item. 

A study-phase retrieval account of the spacing effect instead supposes that when we study a successively presented item, we retrieve memories of the repeated item’s earlier occurrences and their associated contexts. When this happens, it’s proposed that retrieved information is memorally associated with information corresponding to the current presentation context. To put it another way, while the contextual-variability account of the spacing effect emphasizes associating memory of an item with memories of multiple distinct contextual states, the study-phase retrieval account further draws an association between the contextual states themselves, creating an additional possible cue for retrieval whose distinctiveness increases with presentation spacing.

A third account referenced near the end of the \citet{siegel2014retrieved} paper but not realized in retrieved context models like CMR is the deficient processing account. It assumes that a repeated item’s memory trace is encoded (or retrieved) with strength inversely proportional to its lag. Explanations why can vary. \citet{collins2020minerva} for example, focus on an interaction of familiarity and attention in a model of the phenomenon. “Biased exogenous attention” might favor unfamiliar, novel items over familiar ones leaving repetitions of items encoded less thoroughly. Spacing repetitions of items is thought to countervail this effect.

Despite broad success reproducing many repetition effects with CMR, \citet{siegel2014retrieved}'s simulations substantially underpredicted the magnitude of the spacing effect observable in their data. At first glance this gap may be explicable in terms of authors' decision not to fit the model directly to their dataset. However, even when I replicate their simulations using a version of the model with parameters fit directly to their dataset, I found that CMR fails \textt{to the same extent} to identify a large increasing difference in recall probabilities for items with presentations spaced 1-2, 3-5, or 6-8 positions from one another. These results suggest that even as CMR realizes the encoding variability and study-phase retrieval accounts of the spacing effect, it cannot fully account for the full effect observable in behavioral data.

For this project, let's (1) concretize and explain this apparent gap, (2) specify and analyze ways to integrate deficient processing into CMR to enable an improved composite account of spacing and repetition effects in free recall, and (3) try to generate novel predictions about the dynamics of free recall based on how CMR-DE works.