% Options for packages loaded elsewhere
\PassOptionsToPackage{unicode}{hyperref}
\PassOptionsToPackage{hyphens}{url}
\PassOptionsToPackage{dvipsnames,svgnames,x11names}{xcolor}
%
\documentclass[
  letterpaper,
  DIV=11]{article}
\usepackage{amsmath,amssymb}
\usepackage{lmodern}
\usepackage{iftex}
\ifPDFTeX
  \usepackage[T1]{fontenc}
  \usepackage[utf8]{inputenc}
  \usepackage{textcomp} % provide euro and other symbols
\else % if luatex or xetex
  \usepackage{unicode-math}
  \defaultfontfeatures{Scale=MatchLowercase}
  \defaultfontfeatures[\rmfamily]{Ligatures=TeX,Scale=1}
\fi
% Use upquote if available, for straight quotes in verbatim environments
\IfFileExists{upquote.sty}{\usepackage{upquote}}{}
\IfFileExists{microtype.sty}{% use microtype if available
  \usepackage[]{microtype}
  \UseMicrotypeSet[protrusion]{basicmath} % disable protrusion for tt fonts
}{}
\makeatletter
\@ifundefined{KOMAClassName}{% if non-KOMA class
  \IfFileExists{parskip.sty}{%
    \usepackage{parskip}
  }{% else
    \setlength{\parindent}{0pt}
    \setlength{\parskip}{6pt plus 2pt minus 1pt}}
}{% if KOMA class
  \KOMAoptions{parskip=half}}
\makeatother
\usepackage{xcolor}
\IfFileExists{xurl.sty}{\usepackage{xurl}}{} % add URL line breaks if available
\IfFileExists{bookmark.sty}{\usepackage{bookmark}}{\usepackage{hyperref}}
\hypersetup{
  pdftitle={Deficiencies in the Retrieved-Context Account of Spacing and Repetition Effects in Free Recall},
  colorlinks=true,
  linkcolor={blue},
  filecolor={Maroon},
  citecolor={Blue},
  urlcolor={Blue},
  pdfcreator={LaTeX via pandoc}}
\urlstyle{same} % disable monospaced font for URLs
\usepackage{longtable,booktabs,array}
\usepackage{calc} % for calculating minipage widths
% Correct order of tables after \paragraph or \subparagraph
\usepackage{etoolbox}
\makeatletter
\patchcmd\longtable{\par}{\if@noskipsec\mbox{}\fi\par}{}{}
\makeatother
% Allow footnotes in longtable head/foot
\IfFileExists{footnotehyper.sty}{\usepackage{footnotehyper}}{\usepackage{footnote}}
\makesavenoteenv{longtable}
\usepackage{graphicx}
\makeatletter
\def\maxwidth{\ifdim\Gin@nat@width>\linewidth\linewidth\else\Gin@nat@width\fi}
\def\maxheight{\ifdim\Gin@nat@height>\textheight\textheight\else\Gin@nat@height\fi}
\makeatother
% Scale images if necessary, so that they will not overflow the page
% margins by default, and it is still possible to overwrite the defaults
% using explicit options in \includegraphics[width, height, ...]{}
\setkeys{Gin}{width=\maxwidth,height=\maxheight,keepaspectratio}
% Set default figure placement to htbp
\makeatletter
\def\fps@figure{htbp}
\makeatother
\setlength{\emergencystretch}{3em} % prevent overfull lines
\providecommand{\tightlist}{%
  \setlength{\itemsep}{0pt}\setlength{\parskip}{0pt}}
\setcounter{secnumdepth}{5}
\newlength{\cslhangindent}
\setlength{\cslhangindent}{1.5em}
\newlength{\csllabelwidth}
\setlength{\csllabelwidth}{3em}
\newlength{\cslentryspacingunit} % times entry-spacing
\setlength{\cslentryspacingunit}{\parskip}
\newenvironment{CSLReferences}[2] % #1 hanging-ident, #2 entry spacing
 {% don't indent paragraphs
  \setlength{\parindent}{0pt}
  % turn on hanging indent if param 1 is 1
  \ifodd #1
  \let\oldpar\par
  \def\par{\hangindent=\cslhangindent\oldpar}
  \fi
  % set entry spacing
  \setlength{\parskip}{#2\cslentryspacingunit}
 }%
 {}
\usepackage{calc}
\newcommand{\CSLBlock}[1]{#1\hfill\break}
\newcommand{\CSLLeftMargin}[1]{\parbox[t]{\csllabelwidth}{#1}}
\newcommand{\CSLRightInline}[1]{\parbox[t]{\linewidth - \csllabelwidth}{#1}\break}
\newcommand{\CSLIndent}[1]{\hspace{\cslhangindent}#1}
\usepackage{setspace}
\onehalfspacing
%\doublespacing
\usepackage{arxiv}
\makeatletter
\makeatother
\makeatletter
\@ifpackageloaded{caption}{}{\usepackage{caption}}
\AtBeginDocument{%
\renewcommand*\figurename{Figure}
\renewcommand*\tablename{Table}
}
\AtBeginDocument{%
\renewcommand*\listfigurename{List of Figures}
\renewcommand*\listtablename{List of Tables}
}
\@ifpackageloaded{float}{}{\usepackage{float}}
\floatstyle{ruled}
\@ifundefined{c@chapter}{\newfloat{codelisting}{h}{lop}}{\newfloat{codelisting}{h}{lop}[chapter]}
\floatname{codelisting}{Listing}
\newcommand*\listoflistings{\listof{codelisting}{List of Listings}}
\makeatother
\makeatletter
\@ifpackageloaded{caption}{}{\usepackage{caption}}
\@ifpackageloaded{subfig}{}{\usepackage{subfig}}
\makeatother
\ifLuaTeX
  \usepackage{selnolig}  % disable illegal ligatures
\fi

\title{Deficiencies in the Retrieved-Context Account of Spacing and
Repetition Effects in Free Recall}
\author{Jordan B. Gunn\\
Cognition and Cognitive Neuroscience Program\\
Vanderbilt University\\
Nashville, TN 37235\\
\texttt{jordan.gunn@vanderbilt.edu} \and \textbf{Sean M. Polyn}\\
Department of Psychological Sciences\\
Vanderbilt University\\
Nashville, TN 37235\\
\texttt{sean.polyn@vanderbilt.edu}}
\date{}

\begin{document}
\maketitle

Retrieved-context theory (RCT) accounts for free recall of item lists by
asserting that 1) studied items are associated with a gradually
changing, recency-weighted representation of temporal context, 2) the
contexts retrieved from studying (or recalling) items update the current
state of context, and 3) the current state of context is the proximate
cue for each item recall. These mechanisms help account for many
repetition effects in free recall, including the mnemonic advantage of
spaced over massed item repetition and tendencies to successively recall
items that follow a shared repeated item (Lohnas \& Kahana, 2014). Here,
though, we present evidence that this account has significant
deficiencies.

To re-evaluate RCT, we re-analyzed two datasets used to study repetition
effects in free recall (Kahana \& Howard, 2005; Lohnas \& Kahana, 2014).
We applied likelihood-based fitting (Morton \& Polyn, 2016) to a
computational model embodying RCT, the context maintenance and retrieval
model {[}CMR; Polyn, Norman, \& Kahana (2009){]}. Across datasets, the
model underpredicted the mnemonic benefit of spaced over massed
repetition. It also poorly generalized between task conditions that
included or excluded item repetitions in study lists (Busemeyer \& Wang,
2000).

We also report a novel analysis identifying a deficient repetition
contiguity effect: after recalling an item presented repeatedly in a
study list, participants transitioned more often to neighbors of the
item's initial rather than successive presentation(s). CMR poorly
accounts for this pattern, even after adding mechanisms to either reduce
learning for items' successive presentations or enhance memory for
neighbors of items' initial presentations. Contrary to previous reports,
these results suggest that repetition effects may present a significant
challenge for retrieved-context theory.

\hypertarget{introduction}{%
\section{Introduction}\label{introduction}}

First, start with a review of the spacing effect and then of associated
effects.

Then bring up the explanations proposed for it across the literature:
contextual variability, study-phase retrieval, deficient processing.

Review work relating retrieved context theory to two of these accounts,
and outline the main gaps we'll be addressing. The original paper
focused on showing how CMR realizes the contextual variability and
study-phase retrieval effect mechanisms and illustrating similar
patterns in their own dataset. But while the work showed model
mechanisms can generate some benchmark repetition and spacing phenomena,
model fitting was not performed that would evaluate its ability to
account for patterns of retrieval across free recall of lists with item
repetitions. This potentially obscures any hidden inconsistencies in the
retrieved-context account.

Provide high-level overview of how we address this gap in the literature
through model fitting and simulation, novel behavioral analyses,
inspection of model representations, and evaluation of accommodative
mechanisms. Compared to abstract, we can go into enough detail to
clarify what we mean and contextualize our direction with literature.

And a high-level overview of what we found, what we think this means,
and what this meaning would implies.

\hypertarget{methods}{%
\section{Methods}\label{methods}}

How are we evaluating retrieved context theory here? We'll present
datasets, model, and the new-ish evaluation/fitting technique. We may
rename this section (``Evaluating RCT at the Level of the Recall
Sequence'') and/or blend it with the next to emphasize what's new about
our approach additionally use it to review benchmark phenomena in
context of our technique, similarly to how Lynn does in her 2014 paper.
Won't introduce the deficient repetition contiguity effect just yet.

\hypertarget{datasets}{%
\subsection{Datasets}\label{datasets}}

\hypertarget{lohnas2014}{%
\subsubsection{Lohnas2014}\label{lohnas2014}}

\hypertarget{howakaha2005}{%
\subsubsection{HowaKaha2005}\label{howakaha2005}}

\hypertarget{sequence-based-model-evaluation-and-fitting}{%
\subsection{Sequence-Based Model Evaluation and
Fitting}\label{sequence-based-model-evaluation-and-fitting}}

To evaluate how effectively each model accounts for the responses in our
datasets, we applied a likelihood-based model comparison technique
introduced by (\textbf{kragel2015neural?}) that assesses model variants
based on how accurately they can predict the specific sequence in which
items are recalled. According to this method, repeated items and
intrusions (responses naming items not presented in the list) are
included from participants' recall sequences. Given an arbitrary
parameter configuration and sequences of recalls to predict, a model
simulates encoding of each item presented in the corresponding study
list in its respective order. Then, beginning with the first item the
participant recalled in the trial, the probability assigned by the model
to the recall event is recorded. Next, the model simulates retrieval of
that item, and given its updated state is used to similarly predict the
next event in the recall sequence - either retrieval of another item, or
termination of recall - and so on until retrieval terminates. The
probability that the model assigns to each event in the recall sequence
conditional on previous trial events are thus all recorded. These
recorded probabilities are then log-transformed and summed to obtain the
log-likelihood of the entire sequence. Across an entire dataset
containing multiple trials, sequence log-likelihoods can be summed to
obtain a log-likelihood of the entire dataset given the model and its
parameters. Higher log-likelihoods assigned to datasets by a model
correspond to better effectiveness accounting for those datasets.

\hypertarget{cmr}{%
\subsection{CMR}\label{cmr}}

Model specification blended with details of how it realizes repetition
effects.

\hypertarget{references}{%
\section{References}\label{references}}

\hypertarget{refs}{}
\begin{CSLReferences}{1}{0}
\leavevmode\vadjust pre{\hypertarget{ref-busemeyer2000model}{}}%
Busemeyer, J. R., \& Wang, Y.-M. (2000). Model comparisons and model
selections based on generalization criterion methodology. \emph{Journal
of Mathematical Psychology}, \emph{44}(1), 171--189.

\leavevmode\vadjust pre{\hypertarget{ref-kahana2005spacing}{}}%
Kahana, M. J., \& Howard, M. W. (2005). Spacing and lag effects in free
recall of pure lists. \emph{Psychonomic Bulletin \& Review},
\emph{12}(1), 159--164.

\leavevmode\vadjust pre{\hypertarget{ref-lohnas2014retrieved}{}}%
Lohnas, L. L., \& Kahana, M. J. (2014). A retrieved context account of
spacing and repetition effects in free recall. \emph{Journal of
Experimental Psychology: Learning, Memory, and Cognition}, \emph{40}(3),
755.

\leavevmode\vadjust pre{\hypertarget{ref-morton2016predictive}{}}%
Morton, N. W., \& Polyn, S. M. (2016). A predictive framework for
evaluating models of semantic organization in free recall. \emph{Journal
of Memory and Language}, \emph{86}, 119--140.

\leavevmode\vadjust pre{\hypertarget{ref-polyn2009context}{}}%
Polyn, S. M., Norman, K. A., \& Kahana, M. J. (2009). A context
maintenance and retrieval model of organizational processes in free
recall. \emph{Psychological Review}, \emph{116}(1), 129.

\end{CSLReferences}

\end{document}

