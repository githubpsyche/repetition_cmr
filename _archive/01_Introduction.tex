First start with a review of the spacing effect and then of associated effects.

Then bring up the explanations proposed for it across the literature: contextual variability, study-phase retrieval, deficient processing.

Review work relating retrieved context theory to two of these accounts \citep{siegel2014retrieved}, and outline the main gaps we'll be addressing. The original paper focused on showing how CMR realizes the contextual variability and study-phase retrieval effect mechanisms and illustrating similar patterns in their own dataset. It did not attempt to measure how effectively CMR accounts for repetition effects observed in the data through model fitting. As such, it left ambiguous whether other components, such as the deficient processing mechanism proposed by other researches, are necessary to account for the effect.

In this paper, we show that CMR's mechanisms can only partially account for spacing and repetition effects apparent across a range of datasets. Retrieved context theory can effectively account for the monotonic mnemonic effect of spacing when there are short lags between items, but has limited success accounting for the size of the effect as item lags increase further (or something like that -- we'll hash it out). To address these gaps, we outline a way to integrate a deficient processing mechanism into CMR to such that the amount of context retrieved for a repeated item decreases as a function of its lag. Our results suggest an additive role for exogenous attentional processes in the manifestation of spacing and repetition effects in memory.